An audio track in a digital format is represented by a discrete set of values identifying a sound wave. In order to apply machine learning techniques to a track, identifiers for each track is required. With the use of mathematical transformations like the Fourier transformation and its derivaties a collection of values representing each sample is aquired. In this particular study, the feature extraction process has already been applied. Features used are mainly based on MPEG-7 standard descriptors. Features represeting each track are listed below.

\begin{enumerate}
\item Temporal Centroid 
\item Spectral Centroid average value 
\item Audio Spectrum Envelope (ASE) average values in 31 frequency bands
\item ASE average value (averaged for all frequency bands)
\item ASE variance values in 31 frequency bands
\item averaged ASE variance parameters
\item Audio Spectrum Centroid – average and variance values
\item Audio Spectrum Spread – average and variance values
\item Spectral Flatness Measure (SFM) average values for 24 frequency bands
\item SFM average value (averaged for all frequency bands)
\item Spectral Flatness Measure (SFM) variance values for 24 frequency bands
\item Averaged SFM variance parameters
\item 20 first mel cepstral coefficients average values 
\item Harmonic Spectral centroid
\item Harmonic Spectral deviation
\item Harmonic Spectral spread
\item Harmonic Spectral variation
\item Harmonic Ratio average and variance values
\item Upper limit harmonicity average and variance values
\item Log attack time
\end{enumerate}

There parameters sum up to a total of 145 values used to identify track's musical identity. 